\documentclass{article}
\usepackage{hevea}

%----------------------------------------------------------------------
% MACROS

\newcommand{\hsp}{\hspace{0.5in}}
\newcommand{\t}[1]{{\tt #1}}

%----------------------------------------------------------------------
\title{CIL}

\begin{document}
\maketitle

\section{Introduction}

 CIL ({\bf C} {\bf I}ntermediate {\bf L}anguage) is a high-level representation
along with a set of tools that permit easy analysis and source-to-source
transformation of C programs.

 CIL is both lower-level than abstract-syntax trees, by clarifying ambiguous
constructs and removing redundant ones, and also higher-level than typical
intermediate languages designed for compilation, by maintaining types and a
close relationship with the source program. CIL has only a few core constructs
and a syntax-directed type system that makes it easy to analyze and manipulate
C programs, and emit them in a form that can be compiled by a C compiler and
easily correlated with the original source. At the same time we provide a
front-end that translates to CIL not only ANSI C programs but also those using
Microsoft C or GNU C extensions.

 In essence, CIL is a highly-structured, ``clean'' subset of C. CIL features a
reduced number of syntactic and conceptual forms. For example, all looping
constructs are reduced to a single form, all function bodies are given
explicit {\tt return} statements, syntactic sugar like {\tt "->"} is
eliminated and function arguments with array types become pointers. This
reduces the number of cases that must be considered when manipulating a C
program. However, the external representation for CIL will often put back such
syntactic sugar in order to stay faithful to the original source. CIL also
separates type declarations from code and flattens scopes within function
bodies. This structures the program in a manner more amenable to rapid
analysis and transformation. CIL computes the types of all program
expressions, and makes all type promotions and casts explicit. CIL supports
all GCC and MSVC extensions except for nested functions. Finally, CIL
organizes C's imperative features into expressions, instructions and
statements based on the presence and absence of side-effects and control-flow.
Every statement can be annotated with successor and predecessor
information. Thus CIL provides an integrated program representation that can
be used with routines that require an AST (e.g. type-based analyses and
pretty-printers), as well as with routines that require a CFG (e.g., dataflow analyses).

\section{Distribution}

The file \ahref{cil-distrib.tar.gz}{cil-distrib.tar.gz} contains the complete source CIL
distribution, consisting of the following files:

\begin{tabbing}
FilenameXXXXXXXXXXXXXXXXXXXXXXXXX \= DescriptionXXX \kill
\t{Makefile}                    \= Makefile for building CIL \\
\t{Makefile.ocaml}              \= A file that is included by \t{Makefile} \\
\t{doc/}                        \= HTML documentation of the CIL API \\
\t{obj/}                        \= Directory that will contain the compiled
                                   CIL modules and executables\\
\t{src/cil.ml,mli}              \= Definition of CIL abstract syntax and
                                   utilities for manipulating it\\
\t{src/clist.ml,mli}            \= Utilities for efficiently managing lists
                                   that need to be concatenated often\\
\t{src/errormsg.ml,mli}         \= Utilities for error reporting \\
\t{src/frontc/}                 \= The parser and CIL converter \\
\t{src/heapify.ml}              \= A CIL transformation that moves array local
                                   variables from the stack to the heap \\
\t{src/logcalls.ml,mli}         \= A CIL transformation that logs every
                                   function call \\
\t{src/logwrites.ml}            \= A CIL transformation that logs every memory
                                   write \\
\t{src/main.ml}                 \= A test application called \t{cilly} \\
\t{src/pretty.ml,mli}           \= Utilities for pretty printing \\
\t{src/stats.ml,mli}            \= Utilities for maintaining timing statistics
\\
\t{src/trace.ml,mli}            \= Utilities useful for printing debugging
                                   information\\
\t{src/util.ml}                 \= Miscellaneous functions and global variables
\end{tabbing}

\section{Installation}

You will need OCaml release 3.02 or higher to build CIL. CIL has been tested
on Linux and on Windows (where it can behave at either Microsoft Visual C or
gcc). If you want to use CIL on Windows then you must get the source-code
OCaml distribution and compile it yourself using the cygwin tools (as opposed
to getting the Win32 native-code version of OCaml).

\begin{enumerate}
\item Unzip and untar the source distribution. This will create a directory
      called \t{cil-distrib}. \\
      \hsp\verb!gzip -d cil-distrib.tar.gz!\\
      \hsp\verb!tar xvf cil.distrib.tar!
\item Create an environment variable \t{ARCHOS} and set it to either
      \t{X86\_LINUX} or \t{x86\_WIN32}.
\item Enter the \t{cil-distrib} directory and run GNU make to build the 
      distribution.\\
      \hsp\verb!cd cil-distrib!\\
      \hsp\verb!make!\\
      \hsp\verb!make RELEASE=1!\\

\item You can now run either the bytecode or native code version of the
      \t{cilly} test application on a preprocessed C source file. For example,
      to invoke \t{cilly} on \t{foo.i}, do one of the following:\\
      \hsp\verb!obj/cilly.byte.exe foo.i!\\
      \hsp\verb!obj/cilly.asm.exe foo.i!\\
      You can give the option \verb!-help! to get a list of all
      the options accepted by \t{cilly}.
\end{enumerate}

\section{CIL API Documentation} 
Online documentation of the CIL API is available~\ahref{cil-api}{here}. This
documentation is also included in the distribution, in the \t{doc}
subdirectory. 

\section{Credits}

 CIL was develped starting from Hugues Casse's \t{frontc} front-end for C
although all the files from the \t{frontc} distribution have been changed
heavily. 
 
\end{document}



\documentclass{article}
\usepackage{hevea}

\def\t#1{{\tt #1}}
\def\DYNAMIC{\t{DYNAMIC}}
\title{Getting Started}
\author{Scott McPeak \and George Necula}

\begin{document}
\maketitle

 This document is intended to get you started with OCAML and the other tools
that are necessary for CCured and other projects. These tools work on Linux
and Windows (2k, XP and also less reliably on 95/98/Me).

 If you use Linux then you can go directly to Section~\ref{sec-ocaml}.

\section{If you want to use Windows}

 \subsection{Get \t{cygwin}}

 You must have a bunch of Unix tools installed on your machine. (In the future
we might be able to avoid these but for now you are better off with them.).
Here is what I (George) do to install Cygwin. You need a good network
connection for this. 
\begin{itemize}
\item Create a directory \t{C:\backslash Download\backslash cygwin}
\item Go to \ahrefurl{http://sources.redhat.com/cygwin} and click \t{Install
cygwin} icon. Download \t{setup.exe} to the directory you just created.
\item Run \t{setup.exe} and select ``Download to local directory''. Select all
the packages. This will take a while (~ 30 minutes)
\item Run \t{setup.exe} again and not select to ``Install from local
directory''. It is best to {\bf deselect} the \t{tetex} package since I found
it to interfere with other installations of Latex.
\item I choose \t{C:\backslash Programs\backslash cygwin} 
as the home for \t{cygwin}, I use \t{DOS} as the default text file and I
choose ``Install for All''. 
\item Add \t{C:\backslash Programs\backslash cygwin\backslash bin} to your
PATH. You must put it first in the ``System Variables'' PATH (In Control Panel/System/Advanced/Environment
Variables'' and {\bf put it first} so that it comes before the default
c:/WinNT/system32. You can verify that you got it right if you get
\t{/usr/bin/find} when you run \t{which find}. 
\end{itemize}


 \subsection{Customize \t{ssh}}\label{sec-win-ssh}

 Set the environment variable \t{HOME} to point to your home directory. I use
 \t{C:\backslash Necula}. 

 For some strange reason \t{ssh} does not see this \t{HOME} variable and
insists on looking at \t{/home/necula} instead. So I create a link as follows:
\begin{verbatim}
bash
cd /
mkdir home
ln -s /cygdrive/c/Necula /home/necula
\end{verbatim}


\section{Get OCaml}\label{sec-ocaml}

The next step for most people is to download and install the Ocaml
compiler system.  This is available at:

  \ahrefurl{http://caml.inria.fr/ocaml/}

At the time of writing, the current version is 3.02. The following
instructions assume that you will build from sources even on Windows (instead
of using the Native compiler provided). Here the rough sequence of steps:
\begin{itemize}
\item Download and unpack the sources
\item Go in that directory, start \t{bash} if on Windows, and run
\begin{verbatim}
% ./configure
% make world
% make opt
% make install
\end{verbatim}

 The first command might fail to configure some libraries but that's Ok. 
\end{itemize}

To test your ocaml distribution, try:

\begin{verbatim}
  % which ocaml
  /usr/local/bin/ocaml

  % ocaml
        Objective Caml version 3.02

  # exit 0;;    <-- you type "exit 0;;", and press enter
\end{verbatim}


\section{Set the environment variables}

 Make sure that /usr/local/lib in in your path. If you are on Windows this
 means ``C:/Programs/cygwin/usr/local/lib''. 

 Set the \t{ARCHOS} variable to either \t{x86\_WIN32} (if you are on Windows)
 or to \t{x86\_LINUX} (if you are on Linux)


\section{Configure CVS}

 \subsection{\t{.cvsrc}}

 Create a \home{/.cvsrc} file with two lines in it: 
\begin{verbatim}
cvs -q
update -d
\end{verbatim}

 \subsection{Using CVS with \t{ssh}}

 Note: these instructions appear to work even on Windows with the \t{ssh} and
\t{cvs} that ships with \t{cygwin} (provided that you have installed
\t{cygwin} and \t{ssh} as discussed in Section~\ref{sec-win-ssh}).

 Set the environment variable \t{CVS\_RSH} to \t{ssh}. 

 If you want to be able to use \t{ssh} without typing a password everytime
 here is what you can do:

\begin{itemize}
\item If you have an RSA private key that is already authorized on the server,
copy it to \t{\home{}/.ssh/identity} and you should be done.
\item Otherwise
   \begin{enumerate}
     \item Run \t{ssh-keygen} to create a private key. Choose a passphrase and
      remember it. If you do not have a passphrase then anybody who gets
      access to your machine will also be able to log in to the server.
     \item Copy the public key to the server (say manju.cs.berkeley.edu)
\begin{verbatim}
cd ~
scp .ssh/identity.pub manju:~/.ssh/newpublicid
ssh manju
cd .ssh
cat newpublicid >> authorized_keys
\end{verbatim}

   \end{enumerate}

\item If you want you can even start an agent to do the authentication for
you. Documentation to come.
\end{itemize}

 You should read the rest only if you have not used CVS before. 

 CVS is used to synchronize changes to the project across multiple
developers.  See the CVS website for detailed information

  \ahrefurl{http://www.cvshome.org/}
  
There are a few common commands you'll need.  Each of these is to be run
in the base 'cil' directory (the one with 'regrtest'):

\begin{itemize}
\item \t{cvs [-n] update -d [filename]}

    This retrieves any changes recently committed by others.  This is
    usually necessary before you can commit your own changes.  It is a
    good idea to run the fast regression test ('regrtest') before and
    after doing "cvs update" so you can know whether it was you or the
    update which broke something.

    The optional -n flag tells CVS to not actually change any of your
    files.  This is useful for querying the status of the repository.

    The -d argument tells cvs to create on your machine any new directories
    that somebody might have checked in. By default cvs does not create new
    directories. This flag is so useful that many people find it useful to
    create a \home{/.cvsrc} file with one line containing "update -d" in it.
    This way you don't have to specify the flag all the time.

    If you specify a filename (after cd'ing to the directory containing it),
    only that file will be updated, otherwise everything in the current
    directory and below is updated. Run this in the top-level project
    directory to update the entire project. A useful idiom for undoing all of
    your changes is "cd dir; rm file; cvs update file".

    
\item \t{cvs commit [filename]}

    This pushes your changes into the repository, so that the next time
    someone does "cvs update" they will get your changes.  Please try to
    only commit when the regression test script passes.
    
    If you specify a filename, only that file will be committed, otherwise
    everything in the current directory and below is checked in. Run this in
    the top-level project directory to check all of your changes in.

\item  \t{cvs add filename}

    This adds a new file to the repository.  It isn't visible in the
    repository until you do a commit.
\end{itemize}
 

\section{Useful Links}
\begin{itemize}
 \item  Tutorial on ML: 
   \ahrefurl{http://www.dcs.napier.ac.uk/course-notes/sml/manual.html}

        
  \item Documentation and sources for Ocaml: 
        \ahrefurl{http://caml.inria.fr/ocaml/} 


  \item Documentation and sources for CVS:
        \ahrefurl{http://www.cvshome.org/}

  \item Manual for GNU make:
        \ahrefurl{http://www.gnu.org/manual/make/html\_chapter/make\_toc.html}

    
 \end{itemize}

\end{document}

% \subsection{Get \t{cvs}}
%
% At some point the version of \t{cvs} that comes with cygwin was either old or
%had problems with CRLF. A version of \t{cvs} for Windows that I've had success
%with can be found at
%
% \ahrefurl{http://raw.cs.berkeley.edu/cvs.exe}
%
% Copy that file in \t{C:\\Programs\\cygwin\\bin} (thus overwriting the version
%of cvs that comes with cygwin).
%
% \subsection{Get \t{perl}}
%
% Go to \ahrefurl{http://www.activestate.com} and download ActivePerl. Follow
%the instructions. Put the directory that contains the perl executable in your
%PATH. 
%
% \subsection{Get \t{ssh}}
%
%It has been observed that the cygwin version of ssh doesn't behave well
%with cvs because of some CR-LF translation problems. Hence, you will need 
%putty, puttygen, pageant, pscp and plink, all of which are available from:
%
%  \ahrefurl{http://www.chiark.greenend.org.uk/\home{sgtatham}/putty/download.html}
%
%All of the following steps are necessary to get CVS over SSH working:
%\begin{itemize}
%\item Download all components of the Putty suite and put them in a directory.
%Add that directory to the PATH.
%
%\item Run puttygen to create a key pair. Then save the private key in a file,
%say \t{C:/Necula/.ssh/putty\_private}. Copy the public key that puttygen
%displays and save it to a file,  say \t{C:/Necula/.ssh/putty\_public}.
%
%\item Setup pageant (putty ssh-agent): Add a shortcut to the following command
%line (add/change paths as necessary) in your startup folder:
%
%\begin{verbatim}
%      pageant.exe  C:\Necula\.ssh\putty_private
%\end{verbatim}
%
% or equivalently remember to run the above command every time you log in on
%the Windows machine before you try to use cvs. 
%
%\item Add your public key in the "\t{authorized\_keys}" file on the CVS server:
%   One way to do this is to use the following command line:
%
%\begin{verbatim}
%       plink -ssh -pw your-password username@cvs-host "cat >> ~/.ssh/authorized_keys" <  C:\Necula\.ssh\putty_public
%\end{verbatim}
%  
%   This appends your identity.pub file to \t{\home{}/.ssh/authorized\_keys}
%   file on the server. You may do this in any other way you like. You {\bf
%   have} to type your password on the command line if you do it this way,
%   though.
%
%\item Create a "session profile" in putty for the CVS server:
%  \begin{itemize}
%  \item Start putty to see the configuration screen with a category tree on
%   the left 
%  \item Session category: fill in the hostname (brooksie) and protocol (SSH) 
%  \item Connection category: Fill in the auto-login-username field
%  \item Connection/SSH category: Check "allow agent forwarding"
%  \item Session category: Save this session under a name (say, "brooksie")
%  \end{itemize}
%
%
%\item Test your session:
%  \begin{itemize}
%  \item Start pageant if its not already running. It should ask you for a
%    password for your private key (identity). 
%  \item Execute the ls command on the server:
%
%\begin{verbatim}
%             plink @brooksie "ls -a"
%\end{verbatim}
%
%     The command should execute without asking for a password and return to the 
%     windows prompt after executing "ls -a"
%
%   \item Also, the command-line
%\begin{verbatim}
%           putty @brooksie
%\end{verbatim}
%     should log you in directly now.
%  \end{itemize}
%
%\item Set the environment variables on your Windows machine. Note that these
%are different than the usual UNIX settings. The @brooksie part means "use the
%brooksie session from putty"
%
%\begin{verbatim}
%       CVSROOT=:ext:@brooksie:/home/cvs-repository
%       CVS_RSH=plink
%\end{verbatim}
%
%\item At this point you should be able to run cvs:
%
%\begin{verbatim}
%      cvs checkout cil
%\end{verbatim}
%
%\item If you have previously been using cvs in pserver mode, cvs might have
%created CVS/Root files in all subdirectories under its control. This file
%simply stores the value of the CVSROOT environment variable when you had
%checked-out the project for the first time. If CVS finds this file, it seems
%to ignore the environment CVSROOT variable and as a result, might be using
%pserver mode without letting you know.
%   
%   A solution is to delete all the CVS/Root files.
%
%   Another symptom of the CVS/Root problem is the following error message:
%   cvs checkout: warning: unrecognized response `SSH-1.5-1.2.26' from cvs server
%   cvs checkout: warning: unrecognized response `Protocol mismatch.' from cvs	
%   server
%
%   Again, deleting the CVS/Root files should work.
%
%
%(Send corrections/comments/questions to \mailto{ab@amanb.net})
%\end{itemize}
